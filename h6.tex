\documentclass{article}
\usepackage[utf8]{inputenc}
\usepackage[spanish]{babel}
\usepackage{amsthm}
\usepackage{amssymb}
\usepackage{amsmath}
\usepackage{fancyhdr}
\usepackage[dvipsnames, table]{xcolor}
\usepackage[framemethod=tikz]{mdframed}
\usepackage{multicol}
\usepackage{tabularx}
\setlength{\tabcolsep}{3pt}

\definecolor{mycolor}{rgb}{0.122, 0.435, 0.698}

\newmdenv[innerlinewidth=0.5pt, roundcorner=4pt,linecolor=mycolor,innerleftmargin=6pt,
innerrightmargin=6pt,innertopmargin=6pt,innerbottommargin=6pt]{mybox}

\newcommand{\Mod}[1]{\ \hspace{0.2cm} (\mathrm{m\acute{o}d}\ #1)}


\setlength\parindent{0pt}

\newtheorem{definition}{Definición}
\newtheorem{proposition}{Proposición}



\setlength\parindent{0pt}

\renewcommand{\headrulewidth}{0.4pt}

\begin{document}
\date{Jueves 17 de octubre de 2019}
\title{ \textbf{Teoría de números } \\
Hoja de ejercicios 6 }
\author{Docente: Gabriel Chicas Reyes, MSc.\\ 
				Alumno: Kevin López Aquino }
\maketitle	

El siguiente lema será de utilidad para los ejercicios 15 y 16.  \\

\textbf{Lema 1. } Sean $m_{1} , \ldots , m_{r}$ enteros positivos. Entonces, para todo $1 \leq i \leq r$

$$ x \equiv y \Mod{m_{i}} $$

si y solo si $$ x \equiv y \Mod{ \text{ mcm} (m_{1}, \ldots, m_{r} ) }. $$
\begin{proof}
	$(\Rightarrow)$ Si se tiene $x \equiv y \Mod{m_{i}}$ para $1 \leq i \leq r$, se sigue que para cada $m_{i}$, $m_{i} \mid x - y$, de forma que $x - y$ es un múltiplo común de todos los módulos y, por tanto,
	$$ \text{mcm}(m_{1}, \ldots , m_{r}) \mid x - y, $$
	o de forma equivalente, 
	$$ x \equiv y \Mod{\text{mcm}(m_{1}, \ldots , m_{r})}. $$
$(\Leftarrow)$ Supongamos que 

$$ x \equiv y \Mod{\text{mcm}(m_{1}, \ldots , m_{r})}. $$

Entonces podemos deducir cada congruencia $x \equiv y \Mod{m_{i}}$, para $1 \leq i \leq r$, debilitando la congruencia original, puesto que $m_{i} \mid \text{mcm}(m_{1}, \ldots , m_{r}).$

\end{proof}

\newpage

\begin{mybox}
	\textbf{15. } Sea $m = p_{1} \cdot \ldots \cdot p_{r}$ un entero libre de cuadrados. Determine el número de soluciones de la congruencia
	$$ x^{2} \equiv x \Mod{m}. $$ 
\end{mybox}	

$\bullet$ Comenzamos con el siguiente lema. \\

$\textbf{Lema 2. }$ Sea $p$ primo. Entonces, 

$$ x^2 \equiv x \Mod{p}$$

 si y solo si $x \equiv 0 \Mod{p}$ o $x \equiv 1 \Mod{p}$. 
  \begin{proof}
 	$(\Rightarrow)$ Supongamos que 
 	
 	$$ x^2 \equiv x \Mod{p}.$$
 	
 	De esto se sigue que $p \mid x(x - 1)$. Por el lema de Euclides, o bien $p \mid x$, de forma que $x \equiv 0 \Mod{p}$, o bien $p \mid x -1$, de forma que $x \equiv 1 \Mod{p}$. \\
 	
 	$(\Leftarrow)$ Si $x$ es congruente con $0$ o con $1$ módulo $p$, se sigue que $x^{2} \equiv x \Mod{p}$.
 \end{proof}


Procedemos al resultado principal.  \\

\textbf{Proposición. } Sea $m = p_{1} \cdot \ldots \cdot p_{r}$ un entero libre de cuadrados. Entonces, la congruencia
$$ x^{2} \equiv x  \Mod{m}$$
tiene $2^r$ soluciones distintas módulo $m$.
\begin{proof}
	Por el lema anterior, sabemos que para cada primo $p_{i}$ en la factorización de $m$, la congruencia
	$$ x^2 \equiv x \Mod{p_{i}} $$
	implica y es implicada por las dos alternativas
	$$ x \equiv 0 \Mod{p_{i}} \hspace{0.5cm} \text{o} \hspace{0.5cm} x \equiv 1 \Mod{p_{i}} . $$ 
	
Así, para cada $p_{i}$ en la factorización de $m$, hay $2$ posibilidades. Puesto que hay $r$ primos distintos en la factorización de $m$, se sigue que hay $2^r$ formas de construir un sistema de $r$ congruencias. \\
Notando que los módulos en los sistemas de congruencias son coprimos dos a dos, podemos aplicar el teorema chino del resto y deducir que cada sistema de congruencias produce una solución única módulo $p_{1} \cdot \ldots \cdot p_{r} = m.$

\end{proof}

\begin{mybox}
	\textbf{16. } Encontrar todos los idempotentes módulo $2019.$
\end{mybox}	

$\bullet$ Aplicamos la idea de la demostración anterior a este caso concreto. Primero notamos que $2019 = 3 \cdot 673$ y que $3$ y $673$ son primos. Podemos formar los siguientes sistemas de congruencias:

\begin{equation}
\begin{cases} x \equiv 0 \Mod{3} \\ x \equiv 0 \Mod{673} \end{cases}
\end{equation}

\begin{equation}
\begin{cases} x \equiv 1 \Mod{3} \\ x \equiv 1 \Mod{673} \end{cases}
\end{equation}


\begin{equation}
\begin{cases} x \equiv 1 \Mod{3} \\ x \equiv 0 \Mod{673} \end{cases}
\end{equation}


\begin{equation}
\begin{cases} x \equiv 0 \Mod{3} \\ x \equiv 1 \Mod{673} \end{cases}
\end{equation}

Por el teorema chino del resto, cada uno de estos sistemas tiene una única solución módulo $2019$. Por los $\textbf{lemas 1}$ y $\textbf{2}$, si $x$ es la solución de alguno de estos sistemas, se sigue que $x^{2} \equiv x \Mod{2019}$. Además,  estos $4$ sistemas representan todas las soluciones. \\

 La solución al sistema (1) es $x \equiv 0 \Mod{2019}$.  De manera similar, la solución al sistema (2) es $x \equiv 1 \Mod{2019}$. Para el sistema $(3)$, construimos la siguiente tabla, donde $449$ corresponde al inverso multiplicativo de $3$ módulo $673$, encontrado haciendo uso del algoritmo de Euclides. 
	\begin{table}[h]
		\begin{center}
		\begin{tabularx}{0.5\textwidth}{|c |X|X|X|}
			\hline
			\cellcolor{green!25} $m_{j}$  & \cellcolor{green!25} $a_{j}$ & \cellcolor{green!25} $m/m_{j}$ & \cellcolor{green!25} $b_{j}$  \\ \hline
			$3$ & $1$ & $673$ & $1$  \\ \hline
			$673$ & $0$ & $3$ & $449$  \\ \hline
		\end{tabularx}
	\end{center}
	\end{table}	

Por la fórmula del teorema chino del resto, la solución en este caso es,

$$x \equiv 673 \Mod{2019} .$$
Para el sistema $(4)$, la tabla es casi igual:

\begin{table}[h]
	\begin{center}
		\begin{tabularx}{0.5\textwidth}{|c |X|X|X|}
			\hline
			\cellcolor{blue!25} $m_{j}$  & \cellcolor{blue!25} $a_{j}$ & \cellcolor{blue!25} $m/m_{j}$ & \cellcolor{blue!25} $b_{j}$  \\ \hline
			$3$ & $0$ & $673$ & $1$  \\ \hline
			$673$ & $1$ & $3$ & $449$  \\ \hline
		\end{tabularx}
	\end{center}
\end{table}	

De nuevo, por el teorema chino del resto, la solución es

$$x \equiv 3\cdot 449 \equiv 1347 \Mod{2019} . \hspace{0.4cm} \blacksquare $$

\newpage

\begin{mybox}
	\textbf{20. } Sean $m$ y $n$ enteros positivos cualesquiera. Sea $d = \text{mcd}(m, n)$. Demuestre la identiddad
	$$ \varphi(mn) = \frac{d \varphi(m) \varphi(n)}{\varphi(d)} .$$
	¿Qué sucede si $m$ y $n$ son coprimos?
\end{mybox}	

$\bullet$ Notamos que si $m$ y $n$ son coprimos

$$ \text{mcd}(m, n) = 1 $$
$$ \varphi( \text{mcd}(m, n) ) = \varphi(1) = 1 $$

de forma que lo anterior se reduce al hecho que $\varphi(mn) = \varphi(m)\varphi(n)$ cuando $m$ y $n$ son coprimos. De forma más interesante, supongamos que $\varphi(m)\varphi(n) = \varphi(mn)$. Entonces, de la fórmula a demostrar se sigue que 

$$ 1 = \frac{d}{\varphi(d)} $$

$$  \varphi(d) = d .$$

El único número que satisface esta ecuación es $d = 1$. En efecto, si $d \geq 2$, se tiene que $d > \varphi(d)$. De esto se sigue que $m$ y $n$ deben ser coprimos. \\

Veamos un ejemplo. Consideremos $2$ y $10$ y notemos que $\text{mcd}(10, 2) = 2.$ Entonces, 

$$ \varphi(10 \cdot 2) = 10 \cdot 2 \cdot \left( 1 - \frac{1}{2} \right)  \left( 1 - \frac{1}{5} \right)$$

$$ \varphi(10)\varphi(2) = 10 \cdot 2 \cdot  \left( 1 - \frac{1}{2} \right)  \left( 1 - \frac{1}{2} \right)  \left( 1 - \frac{1}{5} \right) $$

$$\varphi(d) = \varphi(2) = 2 \cdot  \left( 1 - \frac{1}{2} \right) $$

Así, se cumple que

$$ \varphi(10 \cdot 2 ) = \frac{2 \cdot  \varphi(10) \cdot \varphi(2)}{ \varphi(2)}. $$

En general,

$$ \frac{\varphi(mn)}{mn} = \prod_{p \mid mn} \left( 1 - \frac{1}{p} \right) = \frac{\frac{1}{m} \prod_{p \mid m} \left( 1 - \frac{1}{p} \right) \cdot \frac{1}{n} \prod_{p \mid n} \left( 1 - \frac{1}{p} \right) }{\frac{1}{\text{mcd}(m, n)} \prod_{p \mid \text{mcd}(m, n)} \left( 1 - \frac{1}{p} \right)} = \frac{\frac{\varphi(m)}{m} \frac{\varphi(n)}{n}}{\frac{\varphi(d)}{d}}.  $$

Así, $\varphi(mn) = \frac{d \varphi(m)\varphi(n)}{\varphi(d)}$. $\blacksquare$

\newpage

\begin{mybox}
	\textbf{21. } Sea $n \geq 2$. Demuestre que la suma de todos los enteros positivos $1 \leq k \leq n$ coprimos con $n$ vale
	
	$$ \frac{1}{2} n \varphi(n) .$$
\end{mybox}	

$ \bullet $ Primero notamos que para todo $1 \leq k \leq n$,

$$ -k \equiv n - k  \Mod{n}$$

Esto implica que $ \text{mcd}(k, n) = \text{mcd}(-k, n) = \text{mcd}(n -k, n) $. De forma que si $k$ es coprimo con $n$, entonces $n - k$ también será coprimo con $n$. \\

Si $n = 2$, la proposición se cumple. Por otro lado, si $n > 2$, en la suma de los coprimos positivos menores que $n$, podemos formar parejas de enteros de la forma
$$ k \hspace{0.5cm} \text{y} \hspace{0.5cm} n - k.$$

Cada pareja suma $n$ y hay $\frac{\varphi(n)}{2}$ parejas, por lo que suma total es

$$ \frac{1}{2} n \varphi(n) . \hspace{0.3cm} \blacksquare$$

\end{document}
